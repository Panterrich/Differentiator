%%% Класс документа
\documentclass[a4paper,14pt]{article}

%%% Работа с русским языком
\usepackage{cmap}					% поиск в PDF
\usepackage[warn]{mathtext}
\usepackage[T2A]{fontenc}			% кодировка
\usepackage[utf8]{inputenc}			% кодировка исходного текста
\usepackage[english,russian]{babel}	% локализация и переносы
\usepackage{mathtext} 				% русские буквы в формулах
\usepackage{csvsimple}              % for tabular from csv loading
\usepackage{indentfirst}            % indent after sections
%\usepackage{minipage}

%%% Дополнительная работа с математикой
\usepackage{amsmath,amsfonts,amssymb,amsthm,mathtools} % AMS
%\usepackage{icomma} % "Умная" запятая: $0,2$ --- число, $0, 2$ --- перечисление
\usepackage{pdflscape}
\usepackage{breqn}

%%% Номера формул
%\mathtoolsset{showonlyrefs=true} % Показывать номера только у тех формул, на которые есть \eqref{} в тексте.
%\usepackage{leqno} % Немуреация формул слева

%%% Шрифты
\usepackage{euscript}	 % Шрифт Евклид
\usepackage{mathrsfs} % Красивый матшрифт

%%% Свои команды
\DeclareMathOperator{\sgn}{\mathop{sgn}}

%%% Перенос знаков в формулах (по Львовскому)
\newcommand*{\hm}[1]{#1\nobreak\discretionary{}
{\hbox{$\mathsurround=0pt #1$}}{}}

%%% Работа с картинками
\usepackage{graphicx}  % Для вставки рисунков
\graphicspath{{images/}{images2/}}  % папки с картинками
\setlength\fboxsep{3pt} % Отступ рамки \fbox{} от рисунка
\setlength\fboxrule{1pt} % Толщина линий рамки \fbox{}
\usepackage{wrapfig} % Обтекание рисунков и таблиц текстом

%%% Работа с таблицами
\usepackage{array,tabularx,tabulary,booktabs} % Дополнительная работа с таблицами
\usepackage{longtable}  % Длинные таблицы
\usepackage{multirow} % Слияние строк в таблице

%%% Теоремы
\theoremstyle{plain} % Это стиль по умолчанию, его можно не переопределять.
%\newtheorem{theorem}{Теорема}[section]
%\newtheorem{proposition}[theorem]{Утверждение}
 
%\theoremstyle{definition} % "Определение"
%\newtheorem{corollary}{Следствие}[theorem]
%\newtheorem{problem}{Задача}[section]
 
%\theoremstyle{remark} % "Примечание"
%\newtheorem*{nonum}{Решение}

%%% Программирование
\usepackage{etoolbox} % логические операторы

%%% Страница
\usepackage{extsizes} % Возможность сделать 14-й шрифт
\usepackage{geometry} % Простой способ задавать поля
	\geometry{top=25mm}
	\geometry{bottom=35mm}
	\geometry{left=35mm}
	\geometry{right=20mm}
	
%%% Колонтитулы
%\usepackage{fancyhdr}
 	%\pagestyle{fancy}
 	%\renewcommand{\headrulewidth}{0mm}  % Толщина линейки, отчеркивающей верхний колонтитул
 	%\lfoot{Нижний левый}
 	%\rfoot{Нижний правый}
 	%\rhead{Верхний правый}
 	%\chead{Верхний в центре}
 	%\lhead{Верхний левый}
 	% \cfoot{Нижний в центре} % По умолчанию здесь номер страницы
 	
%%% Интерлиньяж
%\usepackage{setspace}
%\onehalfspacing % Интерлиньяж 1.5
%\doublespacing % Интерлиньяж 2
%\singlespacing % Интерлиньяж 1

%%% Гиперссылки
\usepackage{hyperref}
\usepackage[usenames,dvipsnames,svgnames,table,rgb]{xcolor}
\hypersetup{				% Гиперссылки
    unicode=true,           % русские буквы в раздела PDF
    pdftitle={Заголовок},   % Заголовок
    pdfauthor={Автор},      % Автор
    pdfsubject={Тема},      % Тема
    pdfcreator={Создатель}, % Создатель
    pdfproducer={Производитель}, % Производитель
    pdfkeywords={keyword1} {key2} {key3}, % Ключевые слова
    colorlinks=true,       	% false: ссылки в рамках; true: цветные ссылки
    linkcolor=red,          % внутренние ссылки
    citecolor=green,        % на библиографию
    filecolor=magenta,      % на файлы
    urlcolor=Mulberry          % на URL
}

%%% Другие пакеты
\usepackage{lastpage} % Узнать, сколько всего страниц в документе.
\usepackage{soul} % Модификаторы начертания
\usepackage{csquotes} % Еще инструменты для ссылок
%\usepackage[style=authoryear,maxcitenames=2,backend=biber,sorting=nty]{biblatex}
\usepackage{multicol} % Несколько колонок
\usepackage{multirow} % Несколько строк

%%% Шрифты
%\renewcommand{\familydefault}{\sfdefault} % Начертание шрифта


%%% Работа с библиографией
%\usepackage{cite} % Работа с библиографией
%\usepackage[superscript]{cite} % Ссылки в верхних индексах
%\usepackage[nocompress]{cite} % 
%\usepackage{csquotes} % Еще инструменты для ссылок


%%% Tikz
\usepackage{tikz} % Работа с графикой
\usepackage{pgfplots} % Работа с pgf
\usepackage{pgfplotstable}

\definecolor{linkcolor}{HTML}{799B03} % цвет ссылок
\definecolor{urlcolor}{HTML}{799B03} % цвет гиперссылок

%%% Дополнительные пакеты для tikz
\usepgfplotslibrary{dateplot} % Возможность подписания дат
\pgfplotsset{compat=1.5}
\usepackage{upgreek}


\begin{document}

\title{Нахождение полной производной сложнейшей функции}
\author{Дурнов А. Н., студент 1 курса ФРТК}
\date{\today}
\maketitle

Все вы знаете, что математический анализ - очень нужная в хозяйстве вещь. Ведь все в нашем мире его знают и активно используют. Засим я пишу эту статью, чтобы убедиться, что все его знают в той степени, что смогут спасти человечество, если наш мир захватит киберпанк. 
\\



Математики вас научат, здесь мы просто верим на слово


\begin{dmath}
\cfrac{\left(x + y\right)^{2}}{y^{2}} \cdot \sh t + \cfrac{\th t}{\arcsin z}
\end{dmath}


Расмотрим все 128 подмножеств нашего множество, включая пустое:


\begin{dmath}
 \cfrac{d}{d x} \left( x \right) = 1 
\end{dmath}


Все физтехи умеют подгонять:


\begin{dmath}
 \cfrac{d}{d x} \left( y \right) = 0 
\end{dmath}


Нет, до четырех штрихов мы не опустимся, согласен даже на крышки


\begin{dmath}
 \cfrac{d}{d x} \left( x + y \right) = 1 + 0 
\end{dmath}


Получается следующая формула:


\begin{dmath}
 \cfrac{d}{d x} \left( \left(x + y\right)^{2} \right) = \left(x + y\right)^{2 - 1} \cdot 2 \cdot \left(1 + 0\right) 
\end{dmath}


Ну здесь я сильно огрубил, мне плевать на эту тройку в знаменателе, мне плевать на этот логарифм в знаменателе, на всё плевать. Я вот так напишу и всё.


\begin{dmath}
 \cfrac{d}{d x} \left( y \right) = 0 
\end{dmath}


Домашнее задание: доказать или опровергнуть гипотезу Римана.


\begin{dmath}
 \cfrac{d}{d x} \left( y^{2} \right) = y^{2 - 1} \cdot 2 \cdot 0 
\end{dmath}


Путин так сказал:


\begin{dmath}
 \cfrac{d}{d x} \left( \cfrac{\left(x + y\right)^{2}}{y^{2}} \right) = \cfrac{\left(x + y\right)^{2 - 1} \cdot 2 \cdot \left(1 + 0\right) \cdot y^{2} - \left(x + y\right)^{2} \cdot y^{2 - 1} \cdot 2 \cdot 0}{\left(y^{2}\right)^{2}} 
\end{dmath}


Вселенная - мир со своими жителями. Дома, вот у них есть планка, выше которой нельзя строить, т.е мы берём и качественно оцениваем хаос. 


\begin{dmath}
 \cfrac{d}{d x} \left( t \right) = 0 
\end{dmath}


Нет, до четырех штрихов мы не опустимся, согласен даже на крышки


\begin{dmath}
 \cfrac{d}{d x} \left( \sh t \right) = \ch t \cdot 0 
\end{dmath}


Расмотрим все 128 подмножеств нашего множество, включая пустое:


\begin{dmath}
 \cfrac{d}{d x} \left( \cfrac{\left(x + y\right)^{2}}{y^{2}} \cdot \sh t \right) = \cfrac{\left(x + y\right)^{2 - 1} \cdot 2 \cdot \left(1 + 0\right) \cdot y^{2} - \left(x + y\right)^{2} \cdot y^{2 - 1} \cdot 2 \cdot 0}{\left(y^{2}\right)^{2}} \cdot \sh t + \cfrac{\left(x + y\right)^{2}}{y^{2}} \cdot \ch t \cdot 0 
\end{dmath}


По 3 теореме Вейрштрасса:


\begin{dmath}
 \cfrac{d}{d x} \left( t \right) = 0 
\end{dmath}


Ну и к бабке не ходи это уже меньше миллиард, даже меньше, чем одна миллиардная, поэтому дальше можно обрубать.


\begin{dmath}
 \cfrac{d}{d x} \left( \th t \right) = \cfrac{0}{\left(\ch t\right)^{2}} 
\end{dmath}


Домашнее задание: доказать или опровергнуть гипотезу Римана.


\begin{dmath}
 \cfrac{d}{d x} \left( z \right) = 0 
\end{dmath}


Ну этот ряд расходится совсем брутально


\begin{dmath}
 \cfrac{d}{d x} \left( \arcsin z \right) = \cfrac{0}{\sqrt{  1 - z^{2}}} 
\end{dmath}


Вот здесь я поделю на 2, потому что...... В общем, потому что могу.


\begin{dmath}
 \cfrac{d}{d x} \left( \cfrac{\th t}{\arcsin z} \right) = \cfrac{\cfrac{0}{\left(\ch t\right)^{2}} \cdot \arcsin z - \th t \cdot \cfrac{0}{\sqrt{  1 - z^{2}}}}{\left(\arcsin z\right)^{2}} 
\end{dmath}


Как можно без запоминания дурацких формул, быстро вспоминать, что такое закон Хука. Ну дело в том, что его звали не Гук, а Хук.


\begin{dmath}
 \cfrac{d}{d x} \left( \cfrac{\left(x + y\right)^{2}}{y^{2}} \cdot \sh t + \cfrac{\th t}{\arcsin z} \right) = \cfrac{\left(x + y\right)^{2 - 1} \cdot 2 \cdot \left(1 + 0\right) \cdot y^{2} - \left(x + y\right)^{2} \cdot y^{2 - 1} \cdot 2 \cdot 0}{\left(y^{2}\right)^{2}} \cdot \sh t + \cfrac{\left(x + y\right)^{2}}{y^{2}} \cdot \ch t \cdot 0 + \cfrac{\cfrac{0}{\left(\ch t\right)^{2}} \cdot \arcsin z - \th t \cdot \cfrac{0}{\sqrt{  1 - z^{2}}}}{\left(\arcsin z\right)^{2}} 
\end{dmath}


Здесь могла быть ваше реклама.


\begin{dmath}
\cfrac{\left(x + y\right)^{2 - 1} \cdot 2 \cdot \left(1 + 0\right) \cdot y^{2} - \left(x + y\right)^{2} \cdot y^{2 - 1} \cdot 2 \cdot 0}{\left(y^{2}\right)^{2}} \cdot \sh t + \cfrac{\left(x + y\right)^{2}}{y^{2}} \cdot \ch t \cdot 0 + \cfrac{\cfrac{0}{\left(\ch t\right)^{2}} \cdot \arcsin z - \th t \cdot \cfrac{0}{\sqrt{  1 - z^{2}}}}{\left(\arcsin z\right)^{2}}
\end{dmath}


По 3 теореме Вейрштрасса:


\begin{dmath}
\cfrac{\left(x + y\right) \cdot 2 \cdot y^{2}}{y^{2 \cdot 2}} \cdot \sh t
\end{dmath}


Получается следующая формула:


\begin{dmath}
\cfrac{\left(x + y\right) \cdot 2 \cdot y^{2}}{y^{4}} \cdot \sh t
\end{dmath}


Я не хочу брать произведение, ну его в болото.


\begin{dmath}
 \cfrac{d}{d y} \left( x \right) = 0 
\end{dmath}


Сарделька - это множество вершин нашего графа, а в нём выбрано подмножество, поэтому оно является подсарделькой:


\begin{dmath}
 \cfrac{d}{d y} \left( y \right) = 1 
\end{dmath}


Редикулус *темная магия*....


\begin{dmath}
 \cfrac{d}{d y} \left( x + y \right) = 0 + 1 
\end{dmath}


Получается следующая формула:


\begin{dmath}
 \cfrac{d}{d y} \left( \left(x + y\right)^{2} \right) = \left(x + y\right)^{2 - 1} \cdot 2 \cdot \left(0 + 1\right) 
\end{dmath}


Формулу разности кубов я не помню, потому что в 7 классе болел


\begin{dmath}
 \cfrac{d}{d y} \left( y \right) = 1 
\end{dmath}


Загадка от Жака Фреско, на размышление даётся 30 сек...


\begin{dmath}
 \cfrac{d}{d y} \left( y^{2} \right) = y^{2 - 1} \cdot 2 \cdot 1 
\end{dmath}


Просто взять вот эту симплициальную резольвенту, взять её абелинизацию и вот окажется, что гомотопические группы абелинизации резольвенты это как-раз таки целочисленные гомологии нашей группы G. Это прекрасно написано в книге Квилена "Гомотопическая алгебра" во второй части.


\begin{dmath}
 \cfrac{d}{d y} \left( \cfrac{\left(x + y\right)^{2}}{y^{2}} \right) = \cfrac{\left(x + y\right)^{2 - 1} \cdot 2 \cdot \left(0 + 1\right) \cdot y^{2} - \left(x + y\right)^{2} \cdot y^{2 - 1} \cdot 2 \cdot 1}{\left(y^{2}\right)^{2}} 
\end{dmath}


Эту надежду надо обязательно проверять. На самом деле надо взять и проверить, что формула такого умножения удовлетворяет всем требуемым свойствам, которые предъявляются к числам: дистрибутвность, коммутативность, ассоциативность. Предоставляю это вам сделать.


\begin{dmath}
 \cfrac{d}{d y} \left( t \right) = 0 
\end{dmath}


Вселенная - мир со своими жителями. Дома, вот у них есть планка, выше которой нельзя строить, т.е мы берём и качественно оцениваем хаос. 


\begin{dmath}
 \cfrac{d}{d y} \left( \sh t \right) = \ch t \cdot 0 
\end{dmath}


Математики вас научат, здесь мы просто верим на слово


\begin{dmath}
 \cfrac{d}{d y} \left( \cfrac{\left(x + y\right)^{2}}{y^{2}} \cdot \sh t \right) = \cfrac{\left(x + y\right)^{2 - 1} \cdot 2 \cdot \left(0 + 1\right) \cdot y^{2} - \left(x + y\right)^{2} \cdot y^{2 - 1} \cdot 2 \cdot 1}{\left(y^{2}\right)^{2}} \cdot \sh t + \cfrac{\left(x + y\right)^{2}}{y^{2}} \cdot \ch t \cdot 0 
\end{dmath}


Я не хочу брать произведение, ну его в болото.


\begin{dmath}
 \cfrac{d}{d y} \left( t \right) = 0 
\end{dmath}


Так поняли, как я это сделал? Ну здесь просто очевидно, просто смотрите и всё сразу видно.


\begin{dmath}
 \cfrac{d}{d y} \left( \th t \right) = \cfrac{0}{\left(\ch t\right)^{2}} 
\end{dmath}


Очевидно, что 


\begin{dmath}
 \cfrac{d}{d y} \left( z \right) = 0 
\end{dmath}


Надоело рассказывать много раз одно и тоже. Щас современный метод расскажу, Султанов поделился.


\begin{dmath}
 \cfrac{d}{d y} \left( \arcsin z \right) = \cfrac{0}{\sqrt{  1 - z^{2}}} 
\end{dmath}


Надоело рассказывать много раз одно и тоже. Щас современный метод расскажу, Султанов поделился.


\begin{dmath}
 \cfrac{d}{d y} \left( \cfrac{\th t}{\arcsin z} \right) = \cfrac{\cfrac{0}{\left(\ch t\right)^{2}} \cdot \arcsin z - \th t \cdot \cfrac{0}{\sqrt{  1 - z^{2}}}}{\left(\arcsin z\right)^{2}} 
\end{dmath}


Ну и к бабке не ходи это уже меньше миллиард, даже меньше, чем одна миллиардная, поэтому дальше можно обрубать.


\begin{dmath}
 \cfrac{d}{d y} \left( \cfrac{\left(x + y\right)^{2}}{y^{2}} \cdot \sh t + \cfrac{\th t}{\arcsin z} \right) = \cfrac{\left(x + y\right)^{2 - 1} \cdot 2 \cdot \left(0 + 1\right) \cdot y^{2} - \left(x + y\right)^{2} \cdot y^{2 - 1} \cdot 2 \cdot 1}{\left(y^{2}\right)^{2}} \cdot \sh t + \cfrac{\left(x + y\right)^{2}}{y^{2}} \cdot \ch t \cdot 0 + \cfrac{\cfrac{0}{\left(\ch t\right)^{2}} \cdot \arcsin z - \th t \cdot \cfrac{0}{\sqrt{  1 - z^{2}}}}{\left(\arcsin z\right)^{2}} 
\end{dmath}


Путин так сказал:


\begin{dmath}
\cfrac{\left(x + y\right)^{2 - 1} \cdot 2 \cdot \left(0 + 1\right) \cdot y^{2} - \left(x + y\right)^{2} \cdot y^{2 - 1} \cdot 2 \cdot 1}{\left(y^{2}\right)^{2}} \cdot \sh t + \cfrac{\left(x + y\right)^{2}}{y^{2}} \cdot \ch t \cdot 0 + \cfrac{\cfrac{0}{\left(\ch t\right)^{2}} \cdot \arcsin z - \th t \cdot \cfrac{0}{\sqrt{  1 - z^{2}}}}{\left(\arcsin z\right)^{2}}
\end{dmath}


Докажем теорему от противного. Предположим противное. Всем противно? Всем противно, теорема доказана.


\begin{dmath}
\cfrac{\left(x + y\right) \cdot 2 \cdot y^{2} - \left(x + y\right)^{2} \cdot y \cdot 2}{y^{2 \cdot 2}} \cdot \sh t
\end{dmath}


Нет, до четырех штрихов мы не опустимся, согласен даже на крышки


\begin{dmath}
\cfrac{\left(x + y\right) \cdot 2 \cdot y^{2} - \left(x + y\right)^{2} \cdot y \cdot 2}{y^{4}} \cdot \sh t
\end{dmath}


Надоело рассказывать много раз одно и тоже. Щас современный метод расскажу, Султанов поделился.


\begin{dmath}
 \cfrac{d}{d t} \left( x \right) = 0 
\end{dmath}


Сарделька - это множество вершин нашего графа, а в нём выбрано подмножество, поэтому оно является подсарделькой:


\begin{dmath}
 \cfrac{d}{d t} \left( y \right) = 0 
\end{dmath}


Квадрат - это не треугольник на стероидах, это отдельная фигура


\begin{dmath}
 \cfrac{d}{d t} \left( x + y \right) = 0 + 0 
\end{dmath}


Домашнее задание: доказать или опровергнуть гипотезу Римана.


\begin{dmath}
 \cfrac{d}{d t} \left( \left(x + y\right)^{2} \right) = \left(x + y\right)^{2 - 1} \cdot 2 \cdot \left(0 + 0\right) 
\end{dmath}


Я, пожалуй, открою схему, потому что здесь без пол-литра не разберёшься.


\begin{dmath}
 \cfrac{d}{d t} \left( y \right) = 0 
\end{dmath}


Из этого следует:


\begin{dmath}
 \cfrac{d}{d t} \left( y^{2} \right) = y^{2 - 1} \cdot 2 \cdot 0 
\end{dmath}


Квадрат - это не треугольник на стероидах, это отдельная фигура


\begin{dmath}
 \cfrac{d}{d t} \left( \cfrac{\left(x + y\right)^{2}}{y^{2}} \right) = \cfrac{\left(x + y\right)^{2 - 1} \cdot 2 \cdot \left(0 + 0\right) \cdot y^{2} - \left(x + y\right)^{2} \cdot y^{2 - 1} \cdot 2 \cdot 0}{\left(y^{2}\right)^{2}} 
\end{dmath}


А кроме того это верно не только для действительного аргумента, но и вообще для любого комплексного.


\begin{dmath}
 \cfrac{d}{d t} \left( t \right) = 1 
\end{dmath}


Просто взять вот эту симплициальную резольвенту, взять её абелинизацию и вот окажется, что гомотопические группы абелинизации резольвенты это как-раз таки целочисленные гомологии нашей группы G. Это прекрасно написано в книге Квилена "Гомотопическая алгебра" во второй части.


\begin{dmath}
 \cfrac{d}{d t} \left( \sh t \right) = \ch t \cdot 1 
\end{dmath}


Ну здесь я сильно огрубил, мне плевать на эту тройку в знаменателе, мне плевать на этот логарифм в знаменателе, на всё плевать. Я вот так напишу и всё.


\begin{dmath}
 \cfrac{d}{d t} \left( \cfrac{\left(x + y\right)^{2}}{y^{2}} \cdot \sh t \right) = \cfrac{\left(x + y\right)^{2 - 1} \cdot 2 \cdot \left(0 + 0\right) \cdot y^{2} - \left(x + y\right)^{2} \cdot y^{2 - 1} \cdot 2 \cdot 0}{\left(y^{2}\right)^{2}} \cdot \sh t + \cfrac{\left(x + y\right)^{2}}{y^{2}} \cdot \ch t \cdot 1 
\end{dmath}


Так поняли, как я это сделал? Ну здесь просто очевидно, просто смотрите и всё сразу видно.


\begin{dmath}
 \cfrac{d}{d t} \left( t \right) = 1 
\end{dmath}


Метод знаменитого нобелевского лауреата Алекса Эдвардсона Султанова:


\begin{dmath}
 \cfrac{d}{d t} \left( \th t \right) = \cfrac{1}{\left(\ch t\right)^{2}} 
\end{dmath}


Просто взять вот эту симплициальную резольвенту, взять её абелинизацию и вот окажется, что гомотопические группы абелинизации резольвенты это как-раз таки целочисленные гомологии нашей группы G. Это прекрасно написано в книге Квилена "Гомотопическая алгебра" во второй части.


\begin{dmath}
 \cfrac{d}{d t} \left( z \right) = 0 
\end{dmath}


Эти методы касаются в основном школьных, ну может университетских, в общем учебных задач. 


\begin{dmath}
 \cfrac{d}{d t} \left( \arcsin z \right) = \cfrac{0}{\sqrt{  1 - z^{2}}} 
\end{dmath}


Я хочу, чтобы каждый из вас получил необходимое и достаточное условие, когда это так:


\begin{dmath}
 \cfrac{d}{d t} \left( \cfrac{\th t}{\arcsin z} \right) = \cfrac{\cfrac{1}{\left(\ch t\right)^{2}} \cdot \arcsin z - \th t \cdot \cfrac{0}{\sqrt{  1 - z^{2}}}}{\left(\arcsin z\right)^{2}} 
\end{dmath}


Квадрат - это не треугольник на стероидах, это отдельная фигура


\begin{dmath}
 \cfrac{d}{d t} \left( \cfrac{\left(x + y\right)^{2}}{y^{2}} \cdot \sh t + \cfrac{\th t}{\arcsin z} \right) = \cfrac{\left(x + y\right)^{2 - 1} \cdot 2 \cdot \left(0 + 0\right) \cdot y^{2} - \left(x + y\right)^{2} \cdot y^{2 - 1} \cdot 2 \cdot 0}{\left(y^{2}\right)^{2}} \cdot \sh t + \cfrac{\left(x + y\right)^{2}}{y^{2}} \cdot \ch t \cdot 1 + \cfrac{\cfrac{1}{\left(\ch t\right)^{2}} \cdot \arcsin z - \th t \cdot \cfrac{0}{\sqrt{  1 - z^{2}}}}{\left(\arcsin z\right)^{2}} 
\end{dmath}


Я, пожалуй, открою схему, потому что здесь без пол-литра не разберёшься.


\begin{dmath}
\cfrac{\left(x + y\right)^{2 - 1} \cdot 2 \cdot \left(0 + 0\right) \cdot y^{2} - \left(x + y\right)^{2} \cdot y^{2 - 1} \cdot 2 \cdot 0}{\left(y^{2}\right)^{2}} \cdot \sh t + \cfrac{\left(x + y\right)^{2}}{y^{2}} \cdot \ch t \cdot 1 + \cfrac{\cfrac{1}{\left(\ch t\right)^{2}} \cdot \arcsin z - \th t \cdot \cfrac{0}{\sqrt{  1 - z^{2}}}}{\left(\arcsin z\right)^{2}}
\end{dmath}


Метод знаменитого нобелевского лауреата Алекса Эдвардсона Султанова:


\begin{dmath}
\cfrac{\left(x + y\right)^{2}}{y^{2}} \cdot \ch t + \cfrac{\cfrac{1}{\left(\ch t\right)^{2}} \cdot \arcsin z}{\left(\arcsin z\right)^{2}}
\end{dmath}


Редикулус *темная магия*....


\begin{dmath}
 \cfrac{d}{d z} \left( x \right) = 0 
\end{dmath}


Как можно без запоминания дурацких формул, быстро вспоминать, что такое закон Хука. Ну дело в том, что его звали не Гук, а Хук.


\begin{dmath}
 \cfrac{d}{d z} \left( y \right) = 0 
\end{dmath}


Следите за руками:


\begin{dmath}
 \cfrac{d}{d z} \left( x + y \right) = 0 + 0 
\end{dmath}


Производная этой части выражена явно через следующее математическое выражение:


\begin{dmath}
 \cfrac{d}{d z} \left( \left(x + y\right)^{2} \right) = \left(x + y\right)^{2 - 1} \cdot 2 \cdot \left(0 + 0\right) 
\end{dmath}


Формулу разности кубов я не помню, потому что в 7 классе болел


\begin{dmath}
 \cfrac{d}{d z} \left( y \right) = 0 
\end{dmath}


Есть такая формула, который знает каждый ребенок, а тот, кто не знает, 2 получает, и розги ему всыпят дома.


\begin{dmath}
 \cfrac{d}{d z} \left( y^{2} \right) = y^{2 - 1} \cdot 2 \cdot 0 
\end{dmath}


Производная этой части выражена явно через следующее математическое выражение:


\begin{dmath}
 \cfrac{d}{d z} \left( \cfrac{\left(x + y\right)^{2}}{y^{2}} \right) = \cfrac{\left(x + y\right)^{2 - 1} \cdot 2 \cdot \left(0 + 0\right) \cdot y^{2} - \left(x + y\right)^{2} \cdot y^{2 - 1} \cdot 2 \cdot 0}{\left(y^{2}\right)^{2}} 
\end{dmath}


Формулу разности кубов я не помню, потому что в 7 классе болел


\begin{dmath}
 \cfrac{d}{d z} \left( t \right) = 0 
\end{dmath}


Домашнее задание: доказать или опровергнуть гипотезу Римана.


\begin{dmath}
 \cfrac{d}{d z} \left( \sh t \right) = \ch t \cdot 0 
\end{dmath}


Редикулус *темная магия*....


\begin{dmath}
 \cfrac{d}{d z} \left( \cfrac{\left(x + y\right)^{2}}{y^{2}} \cdot \sh t \right) = \cfrac{\left(x + y\right)^{2 - 1} \cdot 2 \cdot \left(0 + 0\right) \cdot y^{2} - \left(x + y\right)^{2} \cdot y^{2 - 1} \cdot 2 \cdot 0}{\left(y^{2}\right)^{2}} \cdot \sh t + \cfrac{\left(x + y\right)^{2}}{y^{2}} \cdot \ch t \cdot 0 
\end{dmath}


По 3 теореме Вейрштрасса:


\begin{dmath}
 \cfrac{d}{d z} \left( t \right) = 0 
\end{dmath}


Формулу разности кубов я не помню, потому что в 7 классе болел


\begin{dmath}
 \cfrac{d}{d z} \left( \th t \right) = \cfrac{0}{\left(\ch t\right)^{2}} 
\end{dmath}


Очевидно, что 


\begin{dmath}
 \cfrac{d}{d z} \left( z \right) = 1 
\end{dmath}


Домашнее задание: доказать или опровергнуть гипотезу Римана.


\begin{dmath}
 \cfrac{d}{d z} \left( \arcsin z \right) = \cfrac{1}{\sqrt{  1 - z^{2}}} 
\end{dmath}


Ноль, целковый, полушка, четвертушка...


\begin{dmath}
 \cfrac{d}{d z} \left( \cfrac{\th t}{\arcsin z} \right) = \cfrac{\cfrac{0}{\left(\ch t\right)^{2}} \cdot \arcsin z - \th t \cdot \cfrac{1}{\sqrt{  1 - z^{2}}}}{\left(\arcsin z\right)^{2}} 
\end{dmath}


Есть такая формула, который знает каждый ребенок, а тот, кто не знает, 2 получает, и розги ему всыпят дома.


\begin{dmath}
 \cfrac{d}{d z} \left( \cfrac{\left(x + y\right)^{2}}{y^{2}} \cdot \sh t + \cfrac{\th t}{\arcsin z} \right) = \cfrac{\left(x + y\right)^{2 - 1} \cdot 2 \cdot \left(0 + 0\right) \cdot y^{2} - \left(x + y\right)^{2} \cdot y^{2 - 1} \cdot 2 \cdot 0}{\left(y^{2}\right)^{2}} \cdot \sh t + \cfrac{\left(x + y\right)^{2}}{y^{2}} \cdot \ch t \cdot 0 + \cfrac{\cfrac{0}{\left(\ch t\right)^{2}} \cdot \arcsin z - \th t \cdot \cfrac{1}{\sqrt{  1 - z^{2}}}}{\left(\arcsin z\right)^{2}} 
\end{dmath}


И в этот момент к нам вламываются пифогорейцы и спрашивают: "Где центр сферы, Любовский, где центр сферы?"


\begin{dmath}
\cfrac{\left(x + y\right)^{2 - 1} \cdot 2 \cdot \left(0 + 0\right) \cdot y^{2} - \left(x + y\right)^{2} \cdot y^{2 - 1} \cdot 2 \cdot 0}{\left(y^{2}\right)^{2}} \cdot \sh t + \cfrac{\left(x + y\right)^{2}}{y^{2}} \cdot \ch t \cdot 0 + \cfrac{\cfrac{0}{\left(\ch t\right)^{2}} \cdot \arcsin z - \th t \cdot \cfrac{1}{\sqrt{  1 - z^{2}}}}{\left(\arcsin z\right)^{2}}
\end{dmath}


Загадка от Жака Фреско, на размышление даётся 30 сек...


\begin{dmath}
\cfrac{\th t \cdot \cfrac{1}{\sqrt{  1 - z^{2}}}}{\left(\arcsin z\right)^{2}}
\end{dmath}


И в этот момент к нам вламываются пифогорейцы и спрашивают: "Где центр сферы, Любовский, где центр сферы?"


\begin{dmath}
d f \left(  x,  y,  t,  z \right) = \cfrac{\left(x + y\right) \cdot 2 \cdot y^{2}}{y^{4}} \cdot \sh t \cdot d x  + \cfrac{\left(x + y\right) \cdot 2 \cdot y^{2} - \left(x + y\right)^{2} \cdot y \cdot 2}{y^{4}} \cdot \sh t \cdot d y  +  \left( \cfrac{\left(x + y\right)^{2}}{y^{2}} \cdot \ch t + \cfrac{\cfrac{1}{\left(\ch t\right)^{2}} \cdot \arcsin z}{\left(\arcsin z\right)^{2}} \right)  \cdot d t  + \cfrac{\th t \cdot \cfrac{1}{\sqrt{  1 - z^{2}}}}{\left(\arcsin z\right)^{2}} \cdot d z 
\end{dmath}


Домашнее задание: доказать или опровергнуть гипотезу Римана.


\begin{dmath}
\cfrac{\left(2 + 1\right) \cdot 2 \cdot 1^{2}}{1^{4}} \cdot \sh 4
\end{dmath}


Ноль, целковый, полушка, четвертушка...


\begin{dmath}
6 \cdot \sh 4
\end{dmath}


Один неосторожный syscall - и вы отец.


\begin{dmath}
\cfrac{\left(2 + 1\right) \cdot 2 \cdot 1^{2} - \left(2 + 1\right)^{2} \cdot 1 \cdot 2}{1^{4}} \cdot \sh 4
\end{dmath}


Ну и к бабке не ходи это уже меньше миллиард, даже меньше, чем одна миллиардная, поэтому дальше можно обрубать.


\begin{dmath}
(-12) \cdot \sh 4
\end{dmath}


Доказательство чрезвычайно сложное, я его даже сам не понимаю.


\begin{dmath}
\cfrac{\left(2 + 1\right)^{2}}{1^{2}} \cdot \ch 4 + \cfrac{\cfrac{1}{\left(\ch 4\right)^{2}} \cdot \arcsin 6}{\left(\arcsin 6\right)^{2}}
\end{dmath}


Формула красивая, но бесполезная


\begin{dmath}
9 \cdot \ch 4 + \cfrac{\cfrac{1}{\left(\ch 4\right)^{2}} \cdot \arcsin 6}{\left(\arcsin 6\right)^{2}}
\end{dmath}


Надоело рассказывать много раз одно и тоже. Щас современный метод расскажу, Султанов поделился.


\begin{dmath}
\cfrac{\th 4 \cdot \cfrac{1}{\sqrt{  1 - 6^{2}}}}{\left(\arcsin 6\right)^{2}}
\end{dmath}


Я хочу, чтобы каждый из вас получил необходимое и достаточное условие, когда это так:


\begin{dmath}
\cfrac{\th 4 \cdot \cfrac{1}{\sqrt{  (-35)}}}{\left(\arcsin 6\right)^{2}}
\end{dmath}


Путин так сказал:


\begin{dmath}
d f \left(  2,  1,  4,  6 \right) = 6 \cdot \sh 4 \cdot d x  + (-12) \cdot \sh 4 \cdot d y  +  \left( 9 \cdot \ch 4 + \cfrac{\cfrac{1}{\left(\ch 4\right)^{2}} \cdot \arcsin 6}{\left(\arcsin 6\right)^{2}} \right)  \cdot d t  + \cfrac{\th 4 \cdot \cfrac{1}{\sqrt{  (-35)}}}{\left(\arcsin 6\right)^{2}} \cdot d z 
\end{dmath}

\vspace{2cm}
Вот мы и получили конечный ответ. Я рассказал все методы, которые я знал. Надеюсь это вам поможет для выживания в этом мире. Засим разрешите откланяться.
 

\end{document}
